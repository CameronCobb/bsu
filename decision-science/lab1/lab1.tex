\documentclass[a4paper,14pt]{extreport}

\usepackage[T1,T2A]{fontenc}
\usepackage[utf8]{inputenc}
\usepackage[figure,table]{totalcount}
\usepackage{subcaption}

\usepackage{../packages/bsumain}
\usepackage{../packages/titlepage}
\usepackage{blindtext}

% \newcommand{\quotes}[1]{``#1''}

\jobtitle{Матричные игры}
\labnum{1}

\begin{document}
\maketitle
\setcounter{page}{2}

\chapter{Разрешимость в чистых стратегиях}
\section{Задача 1f}
\subsection{Условие}

Показать, что матричная игра с матрицей $H=(h_{ij})_{nxm}$ имеет решение \par
в чистых стратегиях, и найти такое решение, если: \par 
$n=m$ и для любых $\forall i, j, k$  $1 \leqslant i, j, k \leqslant m$, имеет место тождество: \par
\begin{equation*}
    h_{ij}+h_{jk} + h_{ki} = 0
\end{equation*}

\subsection{Решение}
Найдём верхнее и нижние значения задачи:
\begin{equation}
    \alpha = \max_i\min_jH(i,j)
\end{equation}
\begin{equation}
    \beta = \min_j\max_iH(i,j)
\end{equation}

Учитывая, что $h_{ij}=-h_{jk}-h_{ki}$:
\begin{equation}
\begin{split}
    \alpha = \max_i\min_j(h_{ij}) = \max_i\min_j(-h_{jk} - h_{ki}) \\ = \max_i(-\min_j(h_{jk}) - h_{ki}) = -\min_j(h_{jk}) - \max_i(h_{ki})
\end{split}
\end{equation}
\begin{equation}
\begin{split}
    \beta = \min_j\max_i(h_{ij}) = \min_j\max_i(-h_{jk} - h_{ki}) \\ = \min_j(-h_{jk} -\max_i(h_{ki})) = -\min_j(h_{jk}) - \max_i(h_{ki})
\end{split}
\end{equation}

\begin{equation}
    \alpha (1.3) = \beta (1.4)
\end{equation}
из этого следует, что существует решение в чистых стратегиях.

\subsection{Пример}
\begin{equation*}
    H = \begin{pmatrix} 
            0 & 4 & 2 \\
            -4 & 0 & -2 \\
            -2 & 2 & 0
        \end{pmatrix}
\end{equation*}

Верхние и нижние значения совпадают:
\begin{equation}
    \alpha = \beta = 0
\end{equation}

\section{Задача 2d}
\subsection{Условие}

Найти решение матричной игры с с матрицей $H=(h_{ij})_{nxm}$, если: 
\begin{equation*}
    H = \begin{pmatrix} 
            2 & 4 & 1 & 5 & 1 \\ 
            5 & 2 & 3 & 0 & -1 \\
            2 & -2 & 4 & -3 & 0
        \end{pmatrix}
\end{equation*}

\subsection{Решение}
Найдём минимумы по строкам и максимумы по столбцам:
\begin{equation}
    \alpha_1 = 1, 
    \alpha_2 = -1,
    \alpha_3 = -3, 
\end{equation}
\begin{equation}
    \beta_1 = 5,
    \beta_2 = 4,
    \beta_3 = 4,
    \beta_4 = 5,
    \beta_5 = 1
\end{equation}

Найдём $max(\alpha_i), i=\overline{1, 3}$ и $min(\beta_j), j=\overline{1, 4} $:
\begin{equation}
    \alpha = 1, i_0 = 1
    \beta = 1, j_0 = 5
\end{equation}

Верхнее и нижние значения игры совпадают, \par
следовательно задача разрешима в чистых стратегиях. \par
Оптимальные стратегия для первого - 1, для второго - 5

\chapter{Смешанные стратегии}
\section{Задача 9}
\subsection{Условие}
Проверить, являются ли данные смешанные стратегии и значение игры:
\begin{equation*}
    p = \begin{pmatrix}\frac{1}{4} & 0 & \frac{1}{4} & \frac{1}{2} \end{pmatrix}, 
    q = \begin{pmatrix}\frac{1}{3} & \frac{1}{3} & \frac{1}{3} \end{pmatrix},
    I = 4.
\end{equation*} \par
решением матричной игры с выигрышами:
\begin{equation*}
    H = \begin{pmatrix} 
            14 & -4 & 2 \\
            -4 & 8 & 8 \\
            4 & 4 & 4 \\
            2& 8 & 2
        \end{pmatrix}
\end{equation*}

\subsection{Решение}
Найдём верхние и нижние значения игры по формулe:
\begin{equation}
    \beta(\overline{p}, \overline{q}) = \alpha(\overline{p}, \overline{q}) = \sum_{i=1}^n\sum_{j=1}^mH(i,j) p_i q_j
\end{equation}

\begin{equation}
\begin{split}
    \alpha(p, q) = \beta(p, q) = (14*0.25*0.33) + (-4*0.25*0.33) + (2*0.25*0.33) + \\
    + (-4*0*0.33) + (8*0*0.33) + (8*0*0.33) + \\
    + (4*0.25*0.33) + (4*0.25*0.33) + (4*0.25*0.33) + \\ + (2*0.5*0.33) + (8*0.5*0.33) + (2*0.5*0.33) = 4
\end{split}
\end{equation}

\begin{equation}
    \alpha(p, q) = \beta(p, q) = I
\end{equation}
Поэтому стратегии $p$ и $q$ являются решением матричной игры для $H$

\end{document}