\documentclass[a4paper,14pt]{extreport}

\usepackage[T1,T2A]{fontenc}
\usepackage[utf8]{inputenc}
\usepackage[figure,table]{totalcount}
\usepackage{subcaption}

\usepackage{../packages/bsumain}
\usepackage{../packages/titlepage}
\usepackage{blindtext}

% \newcommand{\quotes}[1]{``#1''}

\jobtitle{Матричные игры}
\labnum{1}

\begin{document}
\maketitle
\setcounter{page}{2}

\chapter{Разрешимость в чистых стратегиях}
\section{Задача 1f}
\subsection{Условие}

Показать, что матричная игра с матрицей $H=(h_{ij})_{nxm}$ имеет решение \par
в чистых стратегиях, и найти такое решение, если: \par 
$n=m$ и для любых $\forall i, j, k$  $1 \leqslant i, j, k \leqslant m$, имеет место тождество: \par
\begin{equation*}
    h_{ij}+h_{jk} + h_{ki} = 0
\end{equation*}

\section{Задача 2d}
\subsection{Условие}

Найти решение матричной игры с с матрицей $H=(h_{ij})_{nxm}$, если: 
\begin{equation*}
    H = \begin{pmatrix} 
            2 & 4 & 1 & 5 & 1 \\ 
            5 & 2 & 3 & 0 & -1 \\
            2 & -2 & 4 & -3 & 0
        \end{pmatrix}
\end{equation*}

\chapter{Смешанные стратегии}
\section{Задача 9}
\subsection{Условие}
Проверить, являются ли данные смешанные стратегии и значение игры:
\begin{equation*}
    p = \begin{pmatrix}\frac{1}{4} & 0 & \frac{1}{4} & \frac{1}{2} \end{pmatrix}, 
    q = \begin{pmatrix}\frac{1}{3} & \frac{1}{3} & \frac{1}{3} \end{pmatrix},
    I = 4.
\end{equation*} \par
решением матричной игры с выигрышами:
\begin{equation*}
    H = \begin{pmatrix} 
            14 & -4 & 2 \\
            -4 & 8 & 8 \\
            4 & 4 & 4 \\
            2& 8 & 2
        \end{pmatrix}
\end{equation*}
\end{document}
