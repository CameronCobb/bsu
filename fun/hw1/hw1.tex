\documentclass{article}
\usepackage[utf8]{inputenc}
\usepackage[T2A]{fontenc}
\usepackage[russian]{babel}

\usepackage{natbib}
\usepackage{graphicx}
\usepackage{amsmath}
\usepackage{xcolor}
\usepackage{amssymb}
\usepackage{graphicx}
\documentclass[oneside, final, 11pt]{article}
\usepackage[paper=a4paper, margin=2cm, bottom=2cm]{geometry}
\begin{titlepage}


\newgeometry{margin=1cm}

\centerline{\large \bf МИНИСТЕРСТВО ОБРАЗОВАНИЯ РЕСПУБЛИКИ БЕЛАРУСЬ}
\bigskip
\bigskip
\centerline{\large \bf БЕЛОРУССКИЙ ГОСУДАРСТВЕННЫЙ УНИВЕРСИТЕТ}
\bigskip
\bigskip
\centerline{\large \bf ФАКУЛЬТЕТ ПРИКЛАДНОЙ МАТЕМАТИКИ И ИНФОРМАТИКИ}
\vfill
\vfill
\vfill
\centerline{\Large \bf ФУНКЦИОНАЛЬНЫЙ АНАЛИЗ}
\bigskip
\bigskip
\vfill
\begin{centering}
  {\large
  Домашняя работа №1 \\
  студента 2 курса 1 группы \\}
\end{centering}
\centerline{\large \bf Пажитных Ивана Павловича}
\vfill
\vfill
\hfill
\begin{minipage}{0.25\textwidth}
  {\large{\bf Преподаватель} \\
{\it Дайняк Виктор \\ Владимирович}}
\end{minipage}
\vfill
\vfill
\centerline{\Large \bf Минск 2016}

\end{titlepage}

\restoregeometry
\begin{document}

\section{№1.8}

\begin{equation*}
    \int_0^1 |x(t)| dt + max_{t\in[0;1]}|x'(t)|
\end{equation*}

\noindent 1)
\begin{aligned}
\begin{equation*}
    \int_0^1 |x(t)| dt + max_{t\in[0;1]}|x'(t)| = 0 \Leftrightarrow x = 0
\end{equation*}
\end{aligned}

\noindent 2)
\begin{aligned}
\begin{equation*}
    \int_0^1 |\alpha x(t)| dt + max_{t\in[0;1]}|\alpha x'(t)| = |\alpha| \left( \int_0^1 |x(t)| dt + max_{t\in[0;1]}|x'(t)| \right)
\end{equation*}
\end{aligned}

\noindent 3)
\begin{aligned}
\begin{equation*}
    \int_0^1 |x(t)+y(t)| dt + max_{t\in[0;1]}|x'(t)+y'(t)| \leq \int_0^1 |x(t)| dt + max_{t\in[0;1]}|x'(t)| + \int_0^1 |y(t)| dt + max_{t\in[0;1]}|y'(t)|
\end{equation*}
\end{aligned}
\\
\\
\\
\section{№2.8}
\begin{equation*}
    x_n(t) = \sqrt[n]{1+t^{2n}}, t \in [0;2]
\end{equation*}

\begin{aligned}
\begin{equation*}
    \forall $ fix $ t \in [0;2] x_n(t) = \sqrt[n]{1+t^{2n}} \xrightarrow[n \to \infty] {} t^2 = a(t) \\
    
    \begin{center}
    ||x_n-a||_{C[0;2]}=max_{t\in[0;2]}|\sqrt[n]{1+t^{2n}}-t^2| \\
    \end{center}
    
    (\sqrt[n]{1+t^{2n}}-t^2)' = \frac{1}{n}(1+t^{2n})^{\frac{1}{n}-1}*2nt^{2n-1}-2t=2t^{2n-1}*(1+t^{2n})^{\frac{1}{n}-1}-2t \\
    
    2t^{2n-1}*(1+t^{2n})^{\frac{1}{n}-1}-2t = 0 \Rightarrow t_1=0, t_2=2 \\
   
    \begin{center}
    |\sqrt[n]{1+t^{2n}}-t^2|_{t=0} = 1 \Rightarrow x_n{t} $ в C[0;2] не сходится к $a(t)=t^2 \\
    \end{center}
\end{equation*}
\end{aligned}
\\
\\
\\
\section{№3.8}
\begin{equation*}
    x_n = \left(\frac{n}{1+n},\frac{n}{1+2n},\ldots,\frac{n}{1+kn},\ldots\right)
\end{equation*}

\begin{aligned}
\begin{equation*}
    x_n = \left(\frac{n}{1+n},\frac{n}{1+2n},\ldots,\frac{n}{1+kn},\ldots\right) = \left(\frac{1}{\frac{1}{n}+1},\frac{1}{\frac{1}{n}+2},\ldots,\frac{1}{\frac{1}{n}+k},\ldots\right) \Rightarrow \\ \Rightarrow x_n $ покординатно сходится к $ a = (1, \frac{1}{2}, \ldots, \frac{1}{k}, \ldots) a \not\in l_5, $ т.к. $ \sum\limits_{i=1}^\infty |\frac{1}{i}|^5 \Rightarrow x_n $ расходится в $ l_5
\end{equation*}
\end{aligned}

\end{document}