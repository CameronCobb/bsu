\documentclass{article}
\usepackage[utf8]{inputenc}
\usepackage[english,russian]{babel}
\usepackage{amssymb,amsthm,amsmath,amscd}
\usepackage{graphicx}
\graphicspath{ {images/} }
\usepackage{float}
\usepackage{listings}
\usepackage{subcaption}

\title{АК - Лабараторная работа 3}
\author{Ivan Pazhitnykh}
\date{December 2016}

\begin{document}

\maketitle
\section{Условие}

\begin{equation}
    \frac{\partial^2u}{\partial x^2} + \frac{\partial^2u}{\partial y^2} = f(x,y)
\end{equation}

\begin{equation*}
\begin{cases}
    u(0,y)=f_1(y)=y^2 \\
    u(a,y)=f_2(y)=sin(y) \\
    u(x,0)=f_3(x)=x^3 \\
    u(x,b)=f_4(x)=x^4  \\
\end{cases}
\end{equation*}

\section{Запуск}
\begin{lstlisting}[language=bash]
    bash run.sh {num_of_procces} {rows} {cols}
\end{lstlisting}

\section{Решение Wolfram}
\begin{figure}[h]
    \centering
    \includegraphics[width=0.8\textwidth]{solving.png}
    \caption{График решения ДУ Пуассона (1)}
    \label{wf}
\end{figure}

\section{Результаты}

\begin{figure}[h]
    \centering
    \begin{subfigure}[h]{0.45\textwidth}
        \includegraphics[width=\textwidth]{1x3x3.png}
        \caption{1 процесс: 3х3}
        \label{s1}
    \end{subfigure}
    ~
    \begin{subfigure}[h]{0.45\textwidth}
        \includegraphics[width=\textwidth]{3x3x3.png}
        \caption{3 процесса: 3х3}
        \label{s2}
    \end{subfigure}
    \caption{Примеры работы}
\end{figure}

\begin{tabular}{|c|c|c|c|c|c|}
\hline
num proc    &   rows    &   cols    &   topo    &   time     & iterations    &   \hline
1           &   40      &   40      &   1D(1x1) &   14.126s  &   1598        &   \hline
2           &   40      &   40      &   2D(1x2) &   7.112s   &   1635        &   \hline
4           &   40      &   40      &   2D(1x4) &   8.605s   &   1670        &   \hline
10          &   40      &   40      &   2D(2x5) &   12.180s  &   1725        &   \hline
\end{tabular} \\

\end{document}