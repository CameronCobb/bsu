\documentclass{article}
\usepackage[utf8]{inputenc}
\usepackage[T2A]{fontenc}
\usepackage[russian]{babel}

\usepackage{natbib}
\usepackage{graphicx}
\usepackage{amsmath}
\usepackage{xcolor}
\usepackage{amssymb}
\usepackage{graphicx}
\documentclass[oneside, final, 11pt]{article}
\usepackage[paper=a4paper, margin=2cm, bottom=2cm]{geometry}
\begin{titlepage}


\newgeometry{margin=1cm}

\centerline{\large \bf МИНИСТЕРСТВО ОБРАЗОВАНИЯ РЕСПУБЛИКИ БЕЛАРУСЬ}
\bigskip
\bigskip
\centerline{\large \bf БЕЛОРУССКИЙ ГОСУДАРСТВЕННЫЙ УНИВЕРСИТЕТ}
\bigskip
\bigskip
\centerline{\large \bf ФАКУЛЬТЕТ ПРИКЛАДНОЙ МАТЕМАТИКИ И ИНФОРМАТИКИ}
\vfill
\vfill
\vfill
\centerline{\Large \bf МЕТОДЫ ЧИСЛЕННОГО АНАЛИЗА}
\bigskip
\bigskip
\vfill
\begin{centering}
  {\large
  Лабараторная работа №3 \\
  студента 2 курса 1 группы \\}
\end{centering}
\centerline{\large \bf Пажитных Ивана Павловича}
\vfill
\vfill
\hfill
\begin{minipage}{0.25\textwidth}
  {\large{\bf Преподаватель} \\
{\it Полещук Максим \\ Александрович}}
\end{minipage}
\vfill
\vfill
\centerline{\Large \bf Минск 2016}

\end{titlepage}

\restoregeometry
\begin{document}

\section{Условие}
В соответствии с вариантом построить кубический сплайн дефекта один для функции $f(x)$, заданной на интервале $[a,b]$ на равномерной сетке с $n+1$ узлами. Значения функции и узлы сетки заданы c тремя значащими цифрами. Сплайн определить в следующем виде:
\begin{equation}
    S_{3,1}^i=a_i+b_i*(x-x_i)+c_i*(x-x_i)^2+d_i*(x-x_i)^3, x\in[x_{i-1},x_i], i=\overline{1, n}
\end{equation}

Для нахождения коэффициентов $c_i$ решить систему линейных уравнений методом Гаусса с выбором главного элемента по столбцу. В качестве граничных условий положить $S_{3,1}''(a) = f''(a), S_{3,1}''(b) = f''(b)$. Вывести коэффициенты всех полиномов сплайна с тремя значащими цифрами, значения абсолютной погрешности $|f(x)-S_{3,1}(x)|$  и оценку $(\frac{b-a}{n})^4*max|f^{(4)}(x)|$ с одной значащей цифрой. Для вычислений использовать тип float.


\section{Вариант}
    \begin{equation}
    x*(3^x + 1)^{-1}, x\in[-2,2], n=5
\end{equation}

\section{Теория}
Будем вычислять значения $a_i, b_i, c_i, d_i$ последовательно по формулам: 

\begin{equation}
\begin{cases}
    c_1 = S^{(2)}(a) = f^{(2)}(a), \\
    h_{i-1}*c_{i-1}+2*(h_{i-1}+h_i)*c_{i}+h_i*c_{i+1} = 3(\frac{y_i-y_{i-1}}{h_i} - \frac{y_{i-1}-y_{i-2}}{h_{i-1}}), i=\overline{1, n-1}, \\
    c_n = S^{(2)}(b) = f^{(2)}(b), $ где $ h_i=x_i-x_{i-1}, y_i = f(x_i) \\
\end{cases}
\end{equation}

\begin{equation}
\begin{cases}
    b_i = \frac{y_i-y_{i-1}}{h_i}-\frac{1}{3}*h_i*(c_{i+1}+2*c_i), i=\overline{1, n-1}, \\
    b_n = \frac{y_n - y_{n-1}}{h_n}-\frac{2}{3}*h_n*c_n
\end{cases}
\end{equation}

\begin{equation}
\begin{cases}
    d_i = \frac{c_{i+1}-c_i}{3h_i} i=\overline{1, n-1}, \\
    d_n = -\frac{c_n}{3h_n}
\end{cases}
\end{equation}

\begin{equation}
    a_i = y_{i-1}, i=\overline{1, n}
\end{equation}

\section{Отчет}

\label{tabular:timesandtenses}
\begin{center}
\begin{tabular}{|c|c|c|c|c|c|c|} \hline
коэффицент: & 1 & 2 & 3 & 4 & 5 & 6 \\ \hline
a & -1.8 & -0.947 & -0.243 & 0.157 & 0.253 & \\
b & 1.07 & 0.879 & 0.5 & 0.121 & -0.0666 & \\
c & 0.0812 & -0.131 & -0.258 & -0.258 & -0.131 & 0.0812 \\
d & -0.0528 & -1.62e-16 & 0.0528 & 0.0886 & -0.0338 & \\ \hline
\end{tabular}
\end{center}

\label{tabular:timesandtenses}
\begin{center}
\begin{tabular}{|c|c|c|c|c|c|} \hline
погрешность: & 1 & 2 & 3 & 4 & 5 \\ \hline
$|f(x)-S_{3,1}(x)|$ & 0.001 & 0.05 & 0.08 & 0.07 & 0.03 \\ \hline
\end{tabular}
\end{center}

\begin{center}
$(\frac{b-a}{n})^4*max|f^{(4)}(x)| = 30.3$
\end{center}

\end{document}
