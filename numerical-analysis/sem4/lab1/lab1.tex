\documentclass{article}
\usepackage[utf8]{inputenc}
\usepackage[T2A]{fontenc}
\usepackage[russian]{babel}

\usepackage{natbib}
\usepackage{graphicx}
\usepackage{amsmath}
\usepackage{xcolor}
\usepackage{amssymb}
\usepackage{graphicx}
\documentclass[oneside, final, 11pt]{article}
\usepackage[paper=a4paper, margin=2cm, bottom=2cm]{geometry}
\begin{titlepage}


\newgeometry{margin=1cm}

\centerline{\large \bf МИНИСТЕРСТВО ОБРАЗОВАНИЯ РЕСПУБЛИКИ БЕЛАРУСЬ}
\bigskip
\bigskip
\centerline{\large \bf БЕЛОРУССКИЙ ГОСУДАРСТВЕННЫЙ УНИВЕРСИТЕТ}
\bigskip
\bigskip
\centerline{\large \bf ФАКУЛЬТЕТ ПРИКЛАДНОЙ МАТЕМАТИКИ И ИНФОРМАТИКИ}
\vfill
\vfill
\vfill
\centerline{\Large \bf МЕТОДЫ ЧИСЛЕННОГО АНАЛИЗА}
\bigskip
\bigskip
\vfill
\begin{centering}
  {\large
  Лабараторная работа №1 \\
  студента 2 курса 1 группы \\}
\end{centering}
\centerline{\large \bf Пажитных Ивана Павловича}
\vfill
\vfill
\hfill
\begin{minipage}{0.25\textwidth}
  {\large{\bf Преподаватель} \\
{\it Полещук Максим \\ Александрович}}
\end{minipage}
\vfill
\vfill
\centerline{\Large \bf Минск 2016}

\end{titlepage}

\restoregeometry

\begin{document}

\section{Условие}
В соответствии с вариантом задания (равным номеру в списке академической группы) произвести табулирование функции , используя чебышевскую сетку с $m$ узлами, $m \in \{3, 4, 5, 6, 8, 10\}$. Для каждого $m$ построить интерполяционный многочлен $P(x)$ Ньютона  на чебышевской сетке.
В отчёте представить значения аргумента $x_n$, приближенные значения функции $P(x_n)$, точные значения функции $f(x_n)$ и оценку погрешности  в  точках исходного отрезка $|r_m(x_n)|$ в $3m$, распределенных равномерно на отрезке $[a,b]$. Вывод всех величин организовать в таблицу. Для каждого $m$ оценить погрешность интерполирования на отрезке (в равномерной норме). Для вычислений использовать тип float.

\section{Вариант}
\begin{equation}
    \label{eq:variant}
    f(x) = x \cdot (3^x + 1)^{-1}, x \in [-2, 2].
\end{equation}

\section{Теория}
Вычислим узлы Чебышева по формулае:
\begin{equation}
    \label{eq:node}
    x_k = \dfrac{a+b}{2} + \dfrac{b-a}{2} \cos{\left(\dfrac{2k+1}{2(n+1)} \pi\right)}, k = 0, 1, \ldots, n. 
    % https://en.wikipedia.org/wiki/Chebyshev_nodes
\end{equation}

\text{По полученным узлам Чебышева заполняется таблица разделённых разностей по следующей формуле:}
\begin{equation}
    \label{eq:raznost_vyvod}
    f(x_i, \ldots, x_{i+k}) = \dfrac{f(x_{i+1}\ldots, x_{i+k}) - f(x_{i+1}\ldots, x_{i+k-1})}{x_{i+k} - x_{i+k-1}}, i = \overline{0, n-1}, k = \overline{1, n-i}.
\end{equation}


Следующая формула представляет собой оценку погрешности интерполяции :
\begin{equation}
    \label{eq:pogr}
    |r_n(x)| \leqslant \dfrac{M_{n+1}}{(n+1)!} \dfrac{(b-a)^{n+1}}{2^{2n + 1}},
\end{equation}
где $M_{n+1} = \underset{[a,b]}{max}|f^{(n+1)}(x)|$, а производная $f^{(n+1)}(x)$ непрерывна на отрезке $[x_0, x_n]$.

Используя полученную таблицу разделённых разонстей строят интерполяционный многочлен Ньютона:
\begin{equation}
    \label{eq:newton}
    P(x) = f(x_0)+\sum\limits_{k=1}^n f(x_0, \ldots, x_k) \cdot (x-x_0) \cdot \ldots \cdot (x-x_{k-1})
\end{equation}

\textbf{Таблица производных исходной функции $f(x) = x \cdot (3^x + 1)^{-1}, x \in [-2, 2]$}
\begin{center}
	\begin{tabular}{| l | l |}
	\hline
	$f(x)$ & $x \cdot (3^x + 1)^{-1}$ \\ \hline
	$f^{(1)}(x)$ & $\dfrac{3^x-3^x x \ln (3)+1}{\left(3^x+1\right)^2}$ \\ \hline
	$f^{(2)}(x)$ & $x \left(\dfrac{2 \cdot 3^{2x} \ln^2(3)}{(3^x+1)^3}-\dfrac{3^x \ln^2(3)}{(3^x+1)^2}\right) - \dfrac{2 \cdot 3^x \ln(3)}{(3^x+1)^2}$ \\ \hline
	\end{tabular}
\end{center}

\section{Отчет}
\subsection{Разделённые разности:}
% $m \in \{3, 4, 5, 6, 8, 10\} \to n \in \{2, 3, 4, 5, 7, 9\}$
for $ m = 3 \Rightarrow n = 2 $.
\begin{center}
\begin{tabular}{|c|c|c|c|c|}
\hline
1.732 &0.2248 &0.1298 &-0.2137\\\hline 
1.225e-16 &6.123e-17 &0.8702 &\\\hline 
-1.732 &-1.507 & &\\\hline 
\end{tabular}
\end{center}

for $ m = 4 \Rightarrow n = 3 $.
\begin{center}
\begin{tabular}{|c|c|c|c|c|c|}
\hline
1.848 &0.2145 &-0.01491 &-0.197 &2.253e-17\\\hline 
0.7654 &0.2306 &0.5 &-0.197 &\\\hline 
-0.7654 &-0.5347 &1.015 & &\\\hline 
-1.848 &-1.633 & & &\\\hline 
\end{tabular}
\end{center}

for $ m = 5 \Rightarrow n = 4 $.
\begin{center}
\begin{tabular}{|c|c|c|c|c|c|c|}
\hline
1.902 &0.2094 &-0.06059 &-0.1452 &0.03143 &0.01652\\\hline 
1.176 &0.2535 &0.2156 &-0.2419 &-0.03143 &\\\hline 
1.225e-16 &6.123e-17 &0.7844 &-0.1452 & &\\\hline 
-1.176 &-0.9221 &1.061 & & &\\\hline 
-1.902 &-1.693 & & & &\\\hline 
\end{tabular}
\end{center}

for $ m = 6 \Rightarrow n = 5 $.
\begin{center}
\begin{tabular}{|c|c|c|c|c|c|c|c|}
\hline
1.932 &0.2066 &-0.07778 &-0.1021 &0.04991 &0.01492 &8.98e-19\\\hline 
1.414 &0.2469 &0.0666 &-0.2243 &-2.944e-17 &0.01492 &\\\hline 
0.5176 &0.1871 &0.5 &-0.2243 &-0.04991 & &\\\hline 
-0.5176 &-0.3305 &0.9334 &-0.1021 & & &\\\hline 
-1.414 &-1.167 &1.078 & & & &\\\hline 
-1.932 &-1.725 & & & & &\\\hline 
\end{tabular}
\end{center}

for $ m = 8 \Rightarrow n = 7 $.
\begin{center}
\begin{tabular}{|c|c|c|c|c|c|c|c|c|c|}
\hline
1.962 &0.2037 &-0.08959 &-0.0571 &0.05299 &0.001613 &-0.005246 &-0.001447 &-8.457e-18\\\hline 
1.663 &0.2305 &-0.04103 &-0.1404 &0.04919 &0.01773 &1.252e-17 &-0.001447 &\\\hline 
1.111 &0.2531 &0.1376 &-0.2414 &-0 &0.01773 &0.005246 & &\\\hline 
0.3902 &0.1539 &0.5 &-0.2414 &-0.04919 &0.001613 & & &\\\hline 
-0.3902 &-0.2363 &0.8624 &-0.1404 &-0.05299 & & & &\\\hline 
-1.111 &-0.858 &1.041 &-0.0571 & & & & &\\\hline 
-1.663 &-1.432 &1.09 & & & & & &\\\hline 
-1.962 &-1.758 & & & & & & &\\\hline 
\end{tabular}
\end{center}

for $ m = 10 \Rightarrow n = 9 $.
\begin{flushleft}
\begin{tabular}{|c|c|c|c|c|c|c|c|c|c|c|c|}
\hline
1.975 &0.2024 &-0.09338 &-0.03848 &0.04561 &-0.006974 &-0.005282 &9.027e-05 &0.0005382 &0.0001432 &-9.948e-19\\\hline 
1.782 &0.2205 &-0.07179 &-0.08716 &0.05721 &0.005113 &-0.005542 &-0.001734 &-2.799e-18 &0.0001432 &\\\hline 
1.414 &0.2469 &0.004392 &-0.1712 &0.0465 &0.02002 &2.576e-17 &-0.001734 &-0.0005382 & &\\\hline 
0.908 &0.2446 &0.1929 &-0.2515 &-6.114e-17 &0.02002 &0.005542 &9.027e-05 & & &\\\hline 
0.3129 &0.1298 &0.5 &-0.2515 &-0.0465 &0.005113 &0.005282 & & & &\\\hline 
-0.3129 &-0.1831 &0.8071 &-0.1712 &-0.05721 &-0.006974 & & & & &\\\hline 
-0.908 &-0.6633 &0.9956 &-0.08716 &-0.04561 & & & & & &\\\hline 
-1.414 &-1.167 &1.072 &-0.03848 & & & & & & &\\\hline 
-1.782 &-1.562 &1.093 & & & & & & & &\\\hline 
-1.975 &-1.773 & & & & & & & & &\\\hline 
\end{tabular}
\end{flushleft}}

\subsection{Оценка погрешности интерполирования:}

for $ m = 3\Rightarrow n = 2 $.
\begin{center}\begin{tabular}{|c|c|c|c|}
\hline
x &f(x) &P(x) &r(x)\\\hline 
-2 &-1.8 &-0.2596 &1.54 \\\hline 
-1.5 &-1.258 &-0.1947 &1.063 \\\hline 
-1 &-0.75 &-0.1298 &0.6202 \\\hline 
-0.5 &-0.317 &-0.06489 &0.2521 \\\hline 
0 &0 &2.776e-17 &2.776e-17 \\\hline 
0.5 &0.183 &0.06489 &0.1181 \\\hline 
1 &0.25 &0.1298 &0.1202 \\\hline 
1.5 &0.2421 &0.1947 &0.04741 \\\hline 
2 &0.2 &0.2596 &0.05957 \\\hline 
\end{tabular}
\end{center}

for $ m = 4\Rightarrow n = 3 $.
\begin{center}\begin{tabular}{|c|c|c|c|}
\hline
x &f(x) &P(x) &r(x)\\\hline 
-2 &-1.8 &-1.825 &0.0248 \\\hline 
-1.636 &-1.404 &-1.382 &0.02137 \\\hline 
-1.273 &-1.021 &-0.9922 &0.02845 \\\hline 
-0.9091 &-0.6644 &-0.654 &0.01037 \\\hline 
-0.5455 &-0.3521 &-0.368 &0.01588 \\\hline 
-0.1818 &-0.09996 &-0.134 &0.03407 \\\hline 
0.1818 &0.08186 &0.04779 &0.03407 \\\hline 
0.5455 &0.1934 &0.1775 &0.01588 \\\hline 
0.9091 &0.2447 &0.2551 &0.01037 \\\hline 
1.273 &0.2521 &0.2806 &0.02845 \\\hline 
1.636 &0.2326 &0.2539 &0.02137 \\\hline 
2 &0.2 &0.1752 &0.0248 \\\hline 
\end{tabular}
\end{center}

for $ m = 5\Rightarrow n = 4 $.
\begin{center}\begin{tabular}{|c|c|c|c|}
\hline
x &f(x) &P(x) &r(x)\\\hline 
-2 &-1.8 &-2.132 &0.3323 \\\hline 
-1.714 &-1.488 &-1.652 &0.164 \\\hline 
-1.429 &-1.182 &-1.238 &0.05516 \\\hline 
-1.143 &-0.8894 &-0.8847 &0.004752 \\\hline 
-0.8571 &-0.6167 &-0.5889 &0.02778 \\\hline 
-0.5714 &-0.3726 &-0.3458 &0.02681 \\\hline 
-0.2857 &-0.1651 &-0.1509 &0.01417 \\\hline 
0 &0 &-8.606e-18 &8.606e-18 \\\hline 
0.2857 &0.1206 &0.1114 &0.009187 \\\hline 
0.5714 &0.1989 &0.1878 &0.0111 \\\hline 
0.8571 &0.2405 &0.2334 &0.007089 \\\hline 
1.143 &0.2534 &0.2527 &0.0006964 \\\hline 
1.429 &0.2461 &0.2501 &0.004004 \\\hline 
1.714 &0.2263 &0.2301 &0.003755 \\\hline 
2 &0.2 &0.1969 &0.003148 \\\hline 
\end{tabular}
\end{center}

for $ m = 6\Rightarrow n = 5 $.
\begin{center}\begin{tabular}{|c|c|c|c|}
\hline
x &f(x) &P(x) &r(x)\\\hline 
-2 &-1.8 &-1.798 &0.002397 \\\hline 
-1.765 &-1.543 &-1.545 &0.002532 \\\hline 
-1.529 &-1.289 &-1.291 &0.001357 \\\hline 
-1.294 &-1.043 &-1.041 &0.001394 \\\hline 
-1.059 &-0.8067 &-0.8037 &0.003071 \\\hline 
-0.8235 &-0.5863 &-0.5836 &0.002727 \\\hline 
-0.5882 &-0.386 &-0.3852 &0.000748 \\\hline 
-0.3529 &-0.2103 &-0.212 &0.001704 \\\hline 
-0.1176 &-0.06262 &-0.06596 &0.00334 \\\hline 
0.1176 &0.05503 &0.05169 &0.00334 \\\hline 
0.3529 &0.1427 &0.141 &0.001704 \\\hline 
0.5882 &0.2023 &0.203 &0.000748 \\\hline 
0.8235 &0.2372 &0.24 &0.002727 \\\hline 
1.059 &0.2521 &0.2552 &0.003071 \\\hline 
1.294 &0.2516 &0.253 &0.001394 \\\hline 
1.529 &0.2402 &0.2389 &0.001357 \\\hline 
1.765 &0.222 &0.2194 &0.002532 \\\hline 
2 &0.2 &0.2024 &0.002397 \\\hline 
\end{tabular}
\end{center}

\begin{center}\begin{tabular}{|c|c|c|c|}
\hline
x &f(x) &P(x) &r(x)\\\hline 
-2 &-1.8 &-1.8 &0.0002371 \\\hline 
-1.826 &-1.61 &-1.609 &0.0002448 \\\hline 
-1.652 &-1.421 &-1.421 &2.035e-05 \\\hline 
-1.478 &-1.235 &-1.235 &0.0002597 \\\hline 
-1.304 &-1.053 &-1.053 &0.0002413 \\\hline 
-1.13 &-0.8771 &-0.8771 &2.841e-05 \\\hline 
-0.9565 &-0.7087 &-0.7085 &0.0002101 \\\hline 
-0.7826 &-0.5499 &-0.5495 &0.0003275 \\\hline 
-0.6087 &-0.4025 &-0.4022 &0.0002653 \\\hline 
-0.4348 &-0.2683 &-0.2683 &6.257e-05 \\\hline 
-0.2609 &-0.149 &-0.1492 &0.0001753 \\\hline 
-0.08696 &-0.04555 &-0.04589 &0.0003316 \\\hline 
0.08696 &0.0414 &0.04107 &0.0003316 \\\hline 
0.2609 &0.1119 &0.1117 &0.0001753 \\\hline 
0.4348 &0.1664 &0.1665 &6.257e-05 \\\hline 
0.6087 &0.2062 &0.2065 &0.0002653 \\\hline 
0.7826 &0.2327 &0.2331 &0.0003275 \\\hline 
0.9565 &0.2478 &0.248 &0.0002101 \\\hline 
1.13 &0.2533 &0.2533 &2.841e-05 \\\hline 
1.304 &0.2513 &0.251 &0.0002413 \\\hline 
1.478 &0.2434 &0.2431 &0.0002597 \\\hline 
1.652 &0.2313 &0.2313 &2.035e-05 \\\hline 
1.826 &0.2165 &0.2167 &0.0002448 \\\hline 
2 &0.2 &0.1998 &0.0002371 \\\hline 
\end{tabular}
\end{center}

for $ m = 10\Rightarrow n = 9 $.
\begin{center}\begin{tabular}{|c|c|c|c|}
\hline
x &f(x) &P(x) &r(x)\\\hline 
-2 &-1.8 &-1.8 &2.352e-05 \\\hline 
-1.862 &-1.649 &-1.649 &2.038e-05 \\\hline 
-1.724 &-1.499 &-1.499 &1.455e-05 \\\hline 
-1.586 &-1.35 &-1.35 &2.584e-05 \\\hline 
-1.448 &-1.203 &-1.203 &6.73e-06 \\\hline 
-1.31 &-1.059 &-1.059 &1.886e-05 \\\hline 
-1.172 &-0.919 &-0.919 &2.998e-05 \\\hline 
-1.034 &-0.7831 &-0.7832 &2.051e-05 \\\hline 
-0.8966 &-0.6528 &-0.6528 &2.04e-06 \\\hline 
-0.7586 &-0.5288 &-0.5288 &2.396e-05 \\\hline 
-0.6207 &-0.4122 &-0.4122 &3.344e-05 \\\hline 
-0.4828 &-0.3039 &-0.3039 &2.596e-05 \\\hline 
-0.3448 &-0.2047 &-0.2047 &5.568e-06 \\\hline 
-0.2069 &-0.1152 &-0.1152 &1.775e-05 \\\hline 
-0.06897 &-0.03579 &-0.03582 &3.294e-05 \\\hline 
0.06897 &0.03318 &0.03314 &3.294e-05 \\\hline 
0.2069 &0.09174 &0.09172 &1.775e-05 \\\hline 
0.3448 &0.1401 &0.1401 &5.568e-06 \\\hline 
0.4828 &0.1788 &0.1789 &2.596e-05 \\\hline 
0.6207 &0.2085 &0.2085 &3.344e-05 \\\hline 
0.7586 &0.2298 &0.2298 &2.396e-05 \\\hline 
0.8966 &0.2438 &0.2438 &2.04e-06 \\\hline 
1.034 &0.2513 &0.2513 &2.051e-05 \\\hline 
1.172 &0.2535 &0.2534 &2.998e-05 \\\hline 
1.31 &0.2511 &0.2511 &1.886e-05 \\\hline 
1.448 &0.2451 &0.2451 &6.73e-06 \\\hline 
1.586 &0.2363 &0.2363 &2.584e-05 \\\hline 
1.724 &0.2255 &0.2255 &1.455e-05 \\\hline 
1.862 &0.2132 &0.2132 &2.038e-05 \\\hline 
2 &0.2 &0.2 &2.352e-05 \\\hline 
\end{tabular}
\end{center}


\end{document}
