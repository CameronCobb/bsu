\documentclass{article}
\usepackage[utf8]{inputenc}
\usepackage[T2A]{fontenc}
\usepackage[russian]{babel}

\usepackage{natbib}
\usepackage{graphicx}
\usepackage{amsmath}
\usepackage{xcolor}
\usepackage{amssymb}
\usepackage{graphicx}
\documentclass[oneside, final, 11pt]{article}
\usepackage[paper=a4paper, margin=2cm, bottom=2cm]{geometry}
\begin{titlepage}


\newgeometry{margin=1cm}

\centerline{\large \bf МИНИСТЕРСТВО ОБРАЗОВАНИЯ РЕСПУБЛИКИ БЕЛАРУСЬ}
\bigskip
\bigskip
\centerline{\large \bf БЕЛОРУССКИЙ ГОСУДАРСТВЕННЫЙ УНИВЕРСИТЕТ}
\bigskip
\bigskip
\centerline{\large \bf ФАКУЛЬТЕТ ПРИКЛАДНОЙ МАТЕМАТИКИ И ИНФОРМАТИКИ}
\vfill
\vfill
\vfill
\centerline{\Large \bf МЕТОДЫ ЧИСЛЕННОГО АНАЛИЗА}
\bigskip
\bigskip
\vfill
\begin{centering}
  {\large
  Лабараторная работа №4 \\
  студента 2 курса 1 группы \\}
\end{centering}
\centerline{\large \bf Пажитных Ивана Павловича}
\vfill
\vfill
\hfill
\begin{minipage}{0.25\textwidth}
  {\large{\bf Преподаватель} \\
{\it Полещук Максим \\ Александрович}}
\end{minipage}
\vfill
\vfill
\centerline{\Large \bf Минск 2016}

\end{titlepage}

\restoregeometry
\begin{document}

\section{Условие}
В соответствии с вариантом, равным номеру в списке академической группы, значения функции $f(x)$, заданной на интервале $[a,b]$, даны в узлах $x_i = a + \frac{b-a}{n}*i, i = \overline{0,n}, n=50$. Значения функции и узлы сетки заданы c тремя значащими цифрами.
Построить квадратичную функцию $P(x) = a_2x^2+a_1x+a_0$, которая даёт для $f(x)$ наилучшее приближение по методу наименьших квадратов. Вывести значения коэффициентов $a_0,a_1,a_2$ с тремя значащими цифрами и среднеквадратичного уклонения с произвольным числом значащих цифр. Для вычислений использовать тип float.

\section{Вариант}
    \begin{equation}
    x*(3^x + 1)^{-1}, x\in[-2,2], n=50
\end{equation}

\section{Теория}
\begin{center}
Положим: $\phi_i = x^i, i \in [0,2]$
\end{center}

Вычисляя скалярное произведение по формуле:
\begin{equation}
    (f, \phi) = \sum\limits_{i=0}^n p(x_i)f(x_i)\phi(x_i)
\end{equation}

Составим систему линейных уравнений относительно коэффицентов $a_i$:
\begin{equation}
    \sum\limits_{i=0}^n a_m(\phi_k, \phi_m) = (f, \phi_k), k \in [0,2]
\end{equation}
Решив её находим многочлен $P(x)$ и считаем среднеквадратичное отклонение по формуле:
\begin{equation}
    r = \sqrt{\frac{1}{n} \sum\limits_{i=0}^n p(x_i)(f(x_i)-P(x_i))^2}
\end{equation}

\section{Отчет}
\begin{center}
    $ a_0 = -0.0256$\\
    $ a_1 = 0.5$\\
    $ a_2 = -0.205$\\
    $ r = 0.020527$
\end{center}

\end{document}
